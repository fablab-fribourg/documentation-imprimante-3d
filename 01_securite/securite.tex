 % !TEX root = ../imprimante3d.tex

\section{Sécurité}

\subsection{Électrocution}

L’alimentation de l’imprimante ainsi que du lit chauffant est en dessous de 50V. Il n’y a donc, normalement, aucun risque d’électrocution.

\subsection{Brulure}

\subsubsection{Tête d'impression}

La tête d’impression est très chaude lors du fonctionnement, de l’ordre de 220 à 270°C . Il faut donc faire attention à ne pas mettre les doigts contre (c'est vite arrivé surtout quand on bidouille).

\subsubsection{Lit chauffant}

Le lit chauffant n'est pas forcément présent sur toutes les machines. Le lit chauffant peut attendre jusqu'à 100°C (même si en général il ne dépasse pas 70°C). Là encore il y a le risque de se brûler. C'est moins grave qu'avec la tête, mais il faut faire attention. Le lit chauffant est comparable à une plaque de cuisson, si on s'en approche, on sent très bien la chaleur.
