 % !TEX root = ../imprimante3d.tex

\section{Impression}

\subsection{Changer le filament}

Si le filament installé est déjà le bon, vous pouvez sauter cette étape. Pour changer le filament, il faut d'abord chauffer la tête d'impression, car le plastique reste collé s'il est froid. Une fois la tête d'impression à $200°C$ ou plus, on peut commencer à retirer délicatement le filament. Pour la Ultimaker, on utilisera la roue dentée de l'extrudeur. Une fois que le filament est bien dehors de la tête, on peut ouvrir l'extrudeur pour libérer le filament. Si le fil part dans tous les sens une fois la bobine à côté, il est préférable de lui mettre du scotch pour le rangement.

\paragraph{} Une fois la nouvelle bobine placée à l'arrière de l'imprimante, il faut enfoncé le câble dans le tube. Si le bout du câble n'est pas proprement coupé, il faut prendre un ciseau ou une pince coupante et couper le bout. On peut enfoncer le fil à la main jusque dans la tête. Ensuite on referme l'extrudeur. On termine par faire avancer gentiment l'extrudeur pour faire sortir de la matière (la tête doit toujours être chaude !). Il est possible qu'il y aie des résidus de l'ancienne matière. Il suffit de faire tourner l'extrudeur à la main jusqu'à ce qu'il n'y en ai plus.

\subsection{Avant l'impression}

Il est possible de préchauffe l'imprimante pour qu'elle soit plus rapidement prête. Le préchauffage est vivement recommandé, surtout si l'imprimante a un lit chauffant. Le lit peut prendre jusqu'à 10 minutes à chauffer, tandis que la tête est prête en 1-2minutes. Le préchauffage peut être fait pendant que nous paramétrons le logiciel par exemple.

 \subsection{Lancer une impression}

Une impression peut être lancée depuis l'ordinateur si l'imprimante est connectée ou depuis le contrôleur si notre pièce est sur une carte SD. Pour le contrôleur, il faut se rentre dans le menu \emph{SC Card}.

\subsection{Pendant l'impression}

Pas grand chose à faire pendant que l'imprimante tourne. Il est bien de venir vérifier de temps en temps que tout se passe bien. Si l'impression foire, ce qui arrive quand même assez fréquemment si on n'est pas très expérimenté, il suffit de stopper l'impression. De corriger le tire et de recommencer. Si vous n'y arriver pas, n'hésitez pas à demander de l'aide !

\subsection{Après l'impression}

Une fois l'impression terminée, la tête d'impression se remet au coin de départ. Dans le cas du lit chauffant, il est impératif d'attendre qu'il refroidisse si vous ne voulez pas plier la pièce ! Une fois froid, le plastique se décolle tout seul. Sans lit chauffant, on peut sans autre enlever la pièce directement en essayant de ne pas trop cassé le scotch.

\paragraph{} Et voilà ! Notre pièce est imprimée !